%==============================================================================
% Paper Research Methods: onderzoeksvoorstel
%==============================================================================

\documentclass{hogent-article}

\usepackage{lipsum} % Voor vultekst

% Invoegen bibliografiebestand
\addbibresource{references.bib}

% Informatie over de opleiding, het vak en soort opdracht
\studyprogramme{Professionele bachelor toegepaste informatica}
\course{Research Methods}
\assignmenttype{Paper: Onderzoeksvoorstel}
\academicyear{2023-2024}

% TODO (fase 1): Werktitel
\title{Werktitel van het voorstel}

% TODO (fase 1): Studentnaam en emailadres invullen
\author{Ernst Aarden}
\email{ernst.aarden@student.hogent.be}

% TODO (fase 1): Medestudent
% Schrijf je het voorstel in samenwerking met een medestudent? Geef dan de naam
% en emailadres hier. Als je het voorstel alleen schrijft, verwijder dan deze
% regels of zet ze in commentaar.
\author{Yasmine Alaoui}
\email{yasmine.alaoui@student.hogent.be}

% TODO (fase 1): Geef hier de link naar jullie Github-repository
\projectrepo{https://github.com/hogenttin/rm-2324-reponame}

% Binnen welke specialisatierichting uit 3TI situeert dit onderzoek zich?
% Kies uit deze lijst:
%
% - Mobile \& Enterprise development
% - AI \& Data Engineering
% - Functional \& Business Analysis
% - System \& Network Administrator
% - Mainframe Expert
% - Als het onderzoek niet past binnen een van deze domeinen specifieer je deze
%   zelf
%
\specialisation{Mobile \& Enterprise development}
% Geef hier enkele sleutelwoorden die je onderwerp beschrijven
\keywords{Scheme, World Wide Web, $\lambda$-calculus}

\begin{document}

\begin{abstract}
  Hier neem je de abstract van je onderzoeksvoorstel over.
\end{abstract}

\tableofcontents

\bigskip

% TODO: EP3
%
% Als je dit voorstel indient in EP3, haal de tekst hieronder uit
% commentaar en pas aan voor jouw situatie.
%
%\paragraph{Opmerking: verbeteringen t.o.v.\ origineel voorstel}
%
% Beschrijf hier kort de verschillen en/of verbeteringen t.o.v. je originele
% voorstel.

% TODO: Bachelorproef
% 
% Neem je dit jaar ook de bachelorproef op? Haal dan de tekst hieronder
% uit commentaar en pas aan voor jouw situatie.
%
%\paragraph{Opmerking}
%
% Ik neem dit jaar ook de bachelorproef op. De inhoud van dit onderzoeksvoorstel dient ook als het onderwerpvoor mijn bachelorproef. Mijn promotor is (Mr./Mevr.) X.\ Familienaam.
%
% Beschrijf de eventuele verschillen en/of verbeteringen in dit document t.o.v.\ jouw onderzoeksvoorstel dat je ingediend hebt voor de bachelorproef.

\section{Inleiding}%
\label{sec:inleiding}

% TODO: (fase 1 - onderzoeksvraag formuleren)

Waarover zal het onderzoek gaan? Introduceer het thema en zorg dat volgende zaken zeker duidelijk aanwezig zijn:

\begin{itemize}
  \item kaderen thema
  \item de doelgroep
  \item de probleemstelling en onderzoeksvraag
  \item de onderzoeksdoelstelling
\end{itemize}

Denk er aan: een typische bachelorproef is \emph{toegepast onderzoek}, wat betekent dat je start vanuit een \emph{concrete casus of probleemsituatie} voor een \emph{specifieke doelgroep} en niet vanuit de gewenste oplossing of technologie die je wilt bespreken.

Het is belangrijk om je onderwerp goed af te bakenen: je gaat \emph{enkel} voor die ene casus op zoek naar een goede oplossing, op basis van de huidige kennis in het vakgebied.

Onder een \emph{specifieke doelgroep} verstaan we typisch één enkel bedrijf of organisatie, of een identificeerbare persoon of groep personen. Dus geen algemene of vaag gedefinieerde groepen zoals \emph{bedrijven} (zelfs niet als je die beperkt tot een bepaalde sector), \emph{developers}, \emph{Vlamingen}, enz. Je richt je in elk geval op it-professionals, een bachelorproef is geen populariserende tekst.

Formuleer duidelijk de onderzoeksvraag! De begeleiders lezen nog steeds te veel voorstellen waarin we geen onderzoeksvraag terugvinden.

Waarom is het nuttig om dit onderwerp te onderzoeken? Wat is de onderzoeksdoelstelling (formuleer deze S.M.A.R.T.)? Wat wil je precies bereiken? Wat zie je als het concrete eindresultaat van je onderzoek, naast de uitgeschreven scriptie? Is het een proof-of-concept, een prototype, een rapport met aanbevelingen, \ldots Met welk eindresultaat kan je je bachelorproef als een succes beschouwen?

\section{Literatuurstudie}%
\label{sec:literatuurstudie}

% TODO: (fase 3, 4 - literatuurstudie)

Hier beschrijf je de \emph{state-of-the-art} rondom je gekozen onderzoeksdomein, d.w.z.\ een inleidende, doorlopende tekst over het onderzoeksdomein van je bachelorproef. Je steunt daarbij heel sterk op de professionele \emph{vakliteratuur}, en niet zozeer op populariserende teksten voor een breed publiek. Wat is de huidige stand van zaken in dit domein, en wat zijn nog eventuele open vragen (die misschien de aanleiding waren tot je onderzoeksvraag!)? 

Je mag deze sectie nog verder onderverdelen in subsecties als dit de structuur van de tekst kan verduidelijken.

Zijn er al gelijkaardige onderzoeken gevoerd? Wat concluderen ze? Wat is het verschil met jouw onderzoek?

Verwijs bij elke introductie van een term of bewering over het domein naar de vakliteratuur! Denk zeker goed na welke werken je refereert en waarom.

Draag zorg voor correcte literatuurverwijzingen! Een bronvermelding hoort thuis \emph{binnen} de zin waar je je op die bron baseert, dus niet er buiten, bijvoorbeeld~\autocite{Hykes2013}! Maak meteen een verwijzing als je gebruik maakt van een bron. Doe dit dus \emph{niet} aan het einde van een lange paragraaf. Baseer nooit teveel aansluitende tekst op eenzelfde bron.

Als je informatie over bronnen verzamelt in JabRef, zorg er dan voor dat alle nodige info aanwezig is om de bron terug te vinden (zoals uitvoerig besproken in de lessen Research Methods). Gebruik bibla\footnote{\url{https://github.com/MrClassicT/bibla}, te installeren met het commando \texttt{pip install bibla}} om je Bib\TeX-be\-stand te controleren op fouten.

% Refereren naar de literatuur kan met:
% \autocite{BIBTEXKEY} => (Auteur, jaartal): voor een referentie tussen
% haakjes, waar de auteursnaam GEEN onderdeel is van een zin.
% \textcite{BIBTEXKEY} => Auteur (jaartal): voor een narratieve referentie,
% waar de naam van de auteur effectief een onderdeel is van de zin.

\section{Methodologie}%
\label{sec:methodologie}

% TODO: (fase 5 - methodologie)

Hier beschrijf je hoe je van plan bent het onderzoek te voeren. Verdeel het onderzoek op in verschillende fasen en probeer te formuleren welke concrete deliverable(s) het resultaat zijn van elke fase.

Welke onderzoekstechniekenen ga je toepassen om elk van je onderzoeksvragen te beantwoorden? Gebruik je hiervoor literatuurstudie, interviews met belanghebbenden (bv.\ voor re\-quire\-ments-a\-na\-ly\-se), experimenten, simulaties, vergelijkende studie, risico-analyse, PoC, \ldots?

Valt je onderwerp onder één van de typische soorten bachelorproeven die besproken zijn in de lessen Research Methods (bv.\ vergelijkende studie of risico-analyse)? Zorg er dan ook voor dat we duidelijk de verschillende stappen terug vinden die we verwachten in dit soort onderzoek!

Pas agile en iteratieve methodes toe en toon dat er een feedbackloop is tussen ontwerp, implementatie en testen. Het is normaal als er een overlap is tussen de verschillende fasen. Meer nog, als alle fasen sequentieel verlopen, is dat een indicatie dat je het \href{https:/commando/en.wikipedia.org/wiki/Waterfall_model}{watervalmodel} hanteert.

Vermijd onderzoekstechnieken die geen objectieve, meetbare resultaten kunnen opleveren. Enquêtes, bijvoorbeeld, zijn voor een bachelorproef informatica meestal \textbf{niet geschikt}. De antwoorden zijn eerder meningen dan feiten en in de praktijk blijkt het ook bijzonder moeilijk om voldoende respondenten te vinden. Studenten die een enquête willen voeren, hebben meestal ook geen goede definitie van de populatie, waardoor ook niet kan aangetoond worden dat eventuele resultaten representatief zijn. Voor het verzamelen van requirements zijn enquêtes of interviews eventueel wel geschikt.

Uit dit onderdeel moet duidelijk naar voor komen dat je bachelorproef ook technisch voldoen\-de diepgang zal bevatten. Het zou niet kloppen als een bachelorproef informatica ook door bv.\ een student marketing zou kunnen uitgevoerd worden.

Je beschrijft ook al welke tools (hardware, software, diensten, \ldots) je denkt hiervoor te gebruiken of te ontwikkelen.

Probeer ook een tijdschatting te maken door een deadline op te geven voor elke fase. Neem voldoende tijd voor de belangrijkste fasen in je onderzoek, nl.\ het uitwerken van je eigen bijdrage (PoC bouwen, experimenten uitvoeren, enz.). Hou er rekening mee dat je typisch één dag per week kan werken aan je bachelorproef. Dat betekent dat uitspraken als ``voor deze fase wordt twee weken tijd voorzien'' erg dubbelzinnig zijn. Betekent dit dat je in realiteit twee dagen zal werken aan deze fase? Of tien werkdagen verspreid over een aantal weken? Zorg dat het duidelijk is wat je precies bedoelt!

% Snippet voor een afbeelding dat je bv kan gebruiken voor een Gantt-diagram.
%
% We gebruiken hier de figure*-omgeving zodat de figuur over beide kolommen
% gespreid wordt voor betere leesbaarheid. Probeer de positionering van
% figuren niet te manipuleren (met bv [ht!]), maar zorg altijd voor een
% zinvol bijschrift en label, en refereer er naar in de tekst.
%
% Bij afbeeldingen die je overneemt, sluit je het bijschrift af met een
% bronvermelding (commando \autocite).
%
% \begin{figure*}
%   \centering
%   \includegraphics[width=\textwidth]{example-image-16x9}
%   \caption{\label{fig:gantt}Gantt diagram met de verschillende fasen en milestones van het onderzoek.}
% \end{figure*}

\section{Verwachte resultaten}%
\label{sec:verwachte-resultaten}

% TODO: (fase 6 - afwerking)

Hier beschrijf je welke resultaten je verwacht en waarom. Bijvoorbeeld, volgens je literatuuronderzoek is softwarepakket A het meest gebruikte en dus denk je dat het voor deze casus ook het meest geschikt zal zijn. Natuurlijk kan je niet in de toekomst kijken en mag je geen alternatieve mogelijkheden uitsluiten.

Als je experimenten, simulaties of metingen uitvoert, kan je overwegen om een mock-up te maken van een grafiek van de uitkomst die je vermoed. Benoem zeker al je assen en meeteenheden die je gaat gebruiken. Hierdoor krijg je ook een concreet beeld van het soort data je zal moeten verzamelen. Pas hierbij toe wat je in Data Science \& AI geleerd hebt over visualisatie van data (bv.\ tonen van spreiding) en toepassen van correcte statistische technieken.

\section{Discussie, verwachte conclusie}%
\label{sec:discussie-conclusie}

Wat heeft de doelgroep van je onderzoek aan het resultaat? Op welke manier biedt jouw onderzoek een meerwaarde?

Het is \textbf{niet} erg indien uit je onderzoek andere resultaten en conclusies vloeien dan dat je hier beschrijft: het is dan juist interessant om te onderzoeken waarom jouw hypothesen niet overeenkomen met de resultaten.

Als je onderwerp zich daartoe leent, kan je eventueel ook suggesties doen voor een vervolg, hetzij verder onderzoek, hetzij verder bouwen op een PoC of prototype tot een eindproduct, hetzij mogelijkheden om de resultaten te valoriseren of commercialiseren.

%------------------------------------------------------------------------------
% Referentielijst
%------------------------------------------------------------------------------
% TODO: (fase 4) de gerefereerde werken moeten in BibTeX-bestand
% bibliografie.bib voorkomen. Gebruik JabRef om je bibliografie bij te
% houden.

\printbibliography[heading=bibintoc]

\end{document}